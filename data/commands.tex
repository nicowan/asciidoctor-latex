% Definition of macros and environments used in asciidoctor document
%


% Command for block marked as [.red] or inline as [red]
\newcommand{\adocMacroRed}[1]{\textcolor{red}{#1}}


% Command for block marked as [.blue] or inline as [blue]
\newcommand{\adocMacroBlue}[1]{\textcolor{blue}{#1}}


% Command for block marked as [.green] or inline as [green]
\newcommand{\adocMacroGreen}[1]{\textcolor{green}{#1}}


% Command for block marked as [.yellow] or inline as [yellow]
\newcommand{\adocMacroYellow}[1]{\textcolor{yellow}{#1}}


% Create a title for a paragraph and other blocks
\newcommand{\adocMacroTitle}[1]{%
  \textbf{\textcolor{black!50!white}{#1}}\newline%
}


% Macro for experimental "menu" command
\newcommand{\adocMacroMenu}[1]{\ovalbox{\texttt{\textit{#1}}}}
\newcommand{\adocMacroNextItem}{\textbf{ $>$ }}


% Macro for experimental "keyboard" command
\newcommand{\adocMacroKbd}[1]{\ovalbox{\texttt{#1}}}


% Macro for experimental "button" command
\newcommand{\adocMacroBtn}[1]{\ovalbox{\textbf{#1}}}


% Defines the adocEnvQuote environment
% Takes 2 parameters : <Title> and <Attribution>
\newcommand{\adocQuoteAttribution}{~}
\newenvironment{adocEnvQuote}[2]{%
    % Placed at environment begin
    \renewcommand{\adocQuoteAttribution}{#2}%
    \begin{quotation}
      \hspace{-7mm}
      {\Huge\bf"}
      \def\temp{#1}\ifx\temp\empty%
        % Nothing here
      \else
        \adocMacroTitle{#1}
      \fi
}{%
  % placed end of environnement

  \hfill\textit{\adocQuoteAttribution}
  \end{quotation}
  \vspace{1em}
}


% Defines the adocEnvVerse environment
% Takes 2 parameters : <Title> and <Attribution>
\newcommand{\adocVerseAttribution}{~}
\newenvironment{adocEnvVerse}[2]{%
    % Placed at environment begin
    \renewcommand{\adocVerseAttribution}{#2}%
    \begin{quotation}
      \hspace{-7mm}
      {\Huge\bf"}
      \def\temp{#1}\ifx\temp\empty%
        % Nothing here
      \else
        \adocMacroTitle{#1}
      \fi
}{%
  % placed end of environnement

  \hfill\textit{\adocVerseAttribution}
  \end{quotation}
  \vspace{1em}
}


% Defines the adocEnvAdmonition environment
% Takes one parameter : <iconname>
\newdimen\admonitionImgWidth{}    % Admonition image width
\newdimen\admonitionTxtWidth{}    % Admonition text width
\newdimen\admonitionTmpWidth{}    % Space between image and text
\newenvironment{adocEnvAdmonition}[1]{
    % Placed at environment begin
    \settowidth{\admonitionImgWidth}{\includegraphics[width=1cm]{#1}}
    \setlength{\admonitionTxtWidth}{-\admonitionImgWidth}
    \settowidth{\admonitionTmpWidth}{~~{\color{black!10}\vrule width 0.1mm}~~~~}
    \setlength{\admonitionTmpWidth}{-\admonitionTmpWidth}
    \addtolength{\admonitionTxtWidth}{\linewidth}
    \addtolength{\admonitionTxtWidth}{\admonitionTmpWidth}
    \begin{minipage}[c]{\admonitionImgWidth}%
        \includegraphics[width=1cm]{#1}
    \end{minipage}%
    ~~{\color{black!10}\vrule width 0.1mm}~~~~
    \begin{minipage}[c]{\admonitionTxtWidth}%
}{
    % placed end of environnement
    \end{minipage}
    \newline
}




% Command to add new image source path to an existing one
%
% https://tex.stackexchange.com/questions/171587/append-entries-to-an-existing-graphicspath
%\makeatletter
%\newcommand\appendtographicspath[1]{%
%  \g@addto@macro\Ginput@path{#1}%
%}
%\makeatother