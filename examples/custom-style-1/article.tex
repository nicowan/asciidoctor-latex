\documentclass[a4paper, 12pt, xcolor=table]{article}

%
% Example of own template
%

% Latex localisation
\usepackage[french]{babel}
\usepackage{fontspec}
    \defaultfontfeatures{Ligatures=TeX}
\usepackage[Latin,Greek]{ucharclasses}

% Manage paper's geometry
\usepackage{pdflscape}
\usepackage{geometry}
    \geometry{
        a4paper,
        left   = 20mm,
        right  = 20mm,
        top    = 20mm,
        bottom = 30mm
    }

% Pages header and footer
\usepackage{fancyhdr}                                                   %
    \newcommand{\doctitle}{}
    \pagestyle{fancy}
    \fancyfoot[C]{~}
    \fancyfoot[R]{\textcolor{red}{\footnotesize\thepage~/~\pageref{LastPage}~}}
    \fancyfoot[L]{\textcolor{blue}{\doctitle}}
    \fancyhead[C]{}
    \fancyhead[L]{}
    \fancyhead[R]{}
    \renewcommand{\headrulewidth}{0cm}
    \renewcommand{\footrulewidth}{0.01cm}

% Paragraph indentation and line spacing
\usepackage{setspace}
    \setstretch{1.1}
    \setlength{\parindent}{0pt}
    \setlength{\parskip}{1ex plus 0.5ex minus 0.2ex}

% Choose the font (only with XeTeX)
\usepackage{lmodern}
%\usepackage[default]{sourcesanspro}
%\usepackage[light]{roboto}
%\usepackage[sfdefault]{overlock}
%\usepackage[defaultfam,extralight,tabular,lining,alternates]{montserrat}
%\usepackage[default,angular]{comicneue}
%\usepackage{cmbright}
%\usepackage[lf]{electrum}

% Colors in the document
\usepackage{color}
\usepackage[usenames,dvipsnames,svgnames,table]{xcolor}
    \definecolor{headcol}{HTML}{202060}

% Links appearance
\usepackage{hyperref}
    \hypersetup{pdfstartview=FitH}
    \hypersetup{colorlinks=true}
    \hypersetup{linkcolor=black!80!blue}
    \hypersetup{anchorcolor=black!80!blue}
    \hypersetup{citecolor=black!80!blue}
    \hypersetup{filecolor=black!80!blue}

% Page references
\usepackage{lastpage}

% Table and array
\usepackage{array}
\usepackage{tabularx}
\usepackage{multirow}
    %\rowcolors{1}{gray!10}{white}
    \newcolumntype{C}[1]{>{\centering\arraybackslash\small}p{#1}}
    \renewcommand{\arraystretch}{1.2}
\usepackage{booktabs} % To thicken table lines

% Draw in documents
\usepackage{tikz}
    \usetikzlibrary{positioning}
    \usetikzlibrary{backgrounds}
    \usetikzlibrary{fit}
    \usetikzlibrary{shapes.geometric}
    \usetikzlibrary{mindmap,trees}
    \usetikzlibrary{arrows,automata}

% Produce nice boxes
\usepackage{fancybox}
\usepackage{framed}
\usepackage[framemethod=TikZ]{mdframed}

% Graphics
\usepackage{wrapfig}
\usepackage{graphicx}
\usepackage{subcaption}
    %% \DeclareGraphicsRule{.tif}{png}{.png}{`convert #1 `dirname #1`/`basename #1 .tif`.png}
    \DeclareGraphicsExtensions{.png, .PNG, .jpg, .JPG, .pdf, .PDF}
    \graphicspath{{./images/}}

\usepackage{tcolorbox}

% To use listings (lstlisting)
\usepackage{listings}

% Using pygment to highlight sources
\usepackage{minted}
    \usemintedstyle{tango}
    \setminted{%
        breaklines,                 % wrap long lines
        fontsize=\footnotesize,     % source code font size
        fontfamily=tt,              % Font type
        linenos=true,               % Display line numbers
        autogobble=true,            % Remove global indentation
        startinline,                % Start in PHP mode for PHP sources
        numbersep=1.5em,
    }

\BeforeBeginEnvironment{minted}{%
    \begin{mdframed}%
    [linecolor=gray!50, backgroundcolor=gray!5!white, roundcorner=5pt]
    %[linecolor=gray!50, backgroundcolor=gray!5!black, roundcorner=5pt]
}

\AfterEndEnvironment{minted}{%
    \end{mdframed}%
}

\newcommand{\inputmintedframed}[2]{%
\begin{mdframed}%
    [linecolor=gray!50, backgroundcolor=gray!5!white, roundcorner=5pt]
    \inputminted{#1}{#2}
\end{mdframed}    
}

% Put a border around verbatim environments and adapt vertical spacing
\BeforeBeginEnvironment{verbatim}{%
    \vspace{-1\baselineskip}
    \begin{mdframed}%
    [linecolor=gray!50, backgroundcolor=gray!5!white, roundcorner=5pt]
    %[linecolor=gray!50, backgroundcolor=gray!5!black, roundcorner=5pt]
}

\AfterEndEnvironment{verbatim}{%
    \end{mdframed}%
    %\vspace{-1\baselineskip}
}



% Package to create our own environments
\usepackage{environ}
